\documentclass{article}
\usepackage{geometry}
\usepackage[utf8]{inputenc}
\usepackage{amssymb}
\usepackage{amsmath}
\usepackage{amsthm}
\usepackage{url}

% bibliography
\PassOptionsToPackage{numbers, compress}{natbib}
\usepackage{natbib}

\newcommand{\reals}{\mathbb{R}}
\newcommand{\normal}{\mathcal{N}}
\newcommand{\pathx}{x}
\newcommand{\pathy}{y}
\newcommand{\measx}{\tilde{x}}
\newcommand{\measy}{\tilde{y}}

\newtheorem{prop}{Proposition}

\title{Efficient Automated Calculation of the Jacobian Determinant for Reversible Jump MCMC}
\author{Marco Cusumano-Towner}
\date{}

\begin{document}

\maketitle

\section{Introduction}
Reversible jump MCMC~\citep{green1995reversible} (RJMCMC) is a general framework for constructing Markov chain Monte Carlo kernels that satisfy detailed balance, while permitting transdimensional and/or deterministic moves.
Some generic transdimensional MCMC kernels that are implemented by various probabilistic programming systems, including `single-site resimulation Metropolis-Hastings' ~\citep{goodman2008church,wingate2011lightweight} can be justified as instances of RJMCMC.
However, it is possible to construct much more efficient MCMC kernels that are specialized to a model using the RJMCMC framework.
The Gen~\citep{cusumano2019gen} probabilistic programming system allows users to program their own custom RJMCMC kernels that are specialized to the model, by specifying an auxiliary generative function called a \emph{proposal} and an \emph{involution} (a function that is its own inverse) that maps tuples of the form $(t, u)$ to tuples of the form $(t', u')$ where $t$ and $t'$ are traces of the model generative function, and $u$ and $u'$ are traces of the proposal generative function.
In general, both $t$ and $u$ (and by necessity $t'$ and $u'$) may contain the values of continuous random choices, in which case the involution must be constructed from a family of differentiable bijections on the values of these continuous random choices.
When applying a particular RJMCMC kernel, the determinant of the Jacobian of one of these bijections is required for the acceptance probability calculation.
This document describes how to implement this Jacobian determinant computation efficiently using automatic differentiation, while exploiting the sparsity of the Jacobian.
The approach uses a domain-specific language (DSL) for specifying families of bijections that makes the sparsity structure of the Jacobian explicit.

\section{Example}

We consider a variant of the changepoint model that was used to introduce RJMCMC in~\citep{green1995reversible}.
Briefly, we put a prior on a class of piecewise constant functions on an interval $[0, x_{max}]$ where the number of changepoints is random and Poisson distributed.
Let $n \in \{1, 2, \ldots, \}$ denote the number of segments, so that the number of changepoints is $n-1 \in \{0, 1, 2, \ldots\}$.
Let $x_i$ denote changepoint $i$, for $i=1,\ldots,n-1$.
Then, segment $1$ has range $[0, x_1)$, segment $d$ has range $[x_{n-1}, x_{max}]$ and segment $i$ has range $[x_{i-1}, x_i)$ for $i=2,\ldots,n-1$.
Let $r_i \in (0, \infty)$ denote the rate for segment $i$ for $i=1,\ldots,n$.
%There is some likelihood function $l_{\mathcal{D}}(n, \mathbf{x}, \mathbf{r})$ where $\mathcal{D}$ denotes some observed data.

Suppose we are performing a birth-death move as described in~\citep{green1995reversible}, in which we either insert a new changepoint or remove an existing changepoint, and adjust the rates.
Let $(n, \mathbf{r}, \mathbf{x})$ denote the the previous hypothesis (corresponding to the previous \emph{trace} of the generative function representing the model; we can omit the observations $\mathcal{D}$ from the trace for to simplify notation because they are held fixed).
The proposal generates the following random choices, dependent on the given trace.
If $n = 1$, then only a birth move is possible, because there are no changepoints to remove.
If $n > 1$, then either a birth or death move is possible and we randomly pick sample $b \in \{0, 1\}$, where $b = 1$ indicates a birth move, from a Bernooulli distribution.
If a birth move is chosen, then the proposal samples an existing segment $i \in \{1, \ldots, n\}$ that will be split into two by the new changepoint from a discrete distribution.
The proposal then samples a value $u_1 \in (x_{i-1}, x_i)$ for the changepoint from a uniform continuous distribution (where $x_{i-1}$ and $x_i$ are the bounds of segment $i$ in the previous trace); this variable represents the locaion of the newly proposed changepoint.
The proposal also samples a value $u_2 \in (0, 1)$ that will be used to determine the new rates $r_{i}'$ and $r_{i+1}'$ from the previous rate $r_i$ and $x_i$ and $u_1$ and $x_{i+1}$ (as described in~\citep{green1995reversible} the new rates are determined such that their harmonic mean, weighted by the size of the segment, is equal to the previous rate).
If a death move is chosen, then the proposal samples an existing changepoint $i \in \{1, \ldots, n-1\}$ to remove.

Suppose the current trace contains $n > 1$ segments, and we are proposing a birth move that is a transition from $n$ segments to $n+1$ segments.
Then, $\mathbf{r} \in \mathbb{R}^n$ and $\mathbf{r}' \in \mathbb{R}^{n+1}$, and $\mathbf{x} \in \mathbb{R}^{n-1}$ and $\mathbf{x'} \in \mathbb{R}^n$.
Specificaly, suppose that $n = 3$ and that $i = 2$ (so we are splitting the middle of three segments).
Then, the differentiable bijection from $(\mathbf{r}, \mathbf{x}, u_1, u_2) \in \mathbb{R}^{7}$ to $(\mathbf{r}', \mathbf{x}') \in \mathbb{R}^{7}$ is:
\[
\begin{array}{rl}
x_1' &= x_1 \\
x_2' &= u_1 \\
x_3' &= x_2 \\
r_1' &= r_1 \\
r_2' &= f_1(r_2, x_1, u_1, x_2, u_2) \\
r_3' &= f_2(r_2, x_1, u_1, x_2, u_2)\\
r_4' &= r_3
\end{array}
\]
for some functions $f_1$ and $f_2$ whose details will not concern us here.
Then, the Jacobian (where columns are inputs and rows are outputs) is as follows, where rows and columns are labeled:
\[
\begin{array}{c|ccccccc}
&       x_1 & x_2 & r_1 &   r_2 &   r_3 &   u_1 &   u_2 \\
\hline
x_1'  & 1   & 0   & 0   &   0   &   0   &   0   &   0   \\
x_2'  & 0   & 0   & 0   &   0   &   0   &   1   &   0   \\
x_3'  & 0   & 1   & 0   &   0   &   0   &   0   &   0   \\
r_1'  & 0   & 0   & 1   &   0   &   0   &   0   &   0   \\
r_2'  & {\scriptstyle \partial f_1 / \partial x_1}
      & {\scriptstyle \partial f_1 / \partial x_2}
      & 0
      & {\scriptstyle \partial f_1 / \partial r_2}
      & 0
      & {\scriptstyle \partial f_1 / \partial u_1}
      & {\scriptstyle \partial f_1 / \partial u_2}   \\
r_3'  & {\scriptstyle \partial f_2 / \partial x_1}
      & {\scriptstyle \partial f_2 / \partial x_2}
      & 0
      & {\scriptstyle \partial f_2 / \partial r_2}
      & 0
      & {\scriptstyle \partial f_2 / \partial u_1}
      & {\scriptstyle \partial f_2 / \partial u_2}   \\
r_4'  & 0   & 0   & 0   &   0   &   1   &   0   &   0
\end{array}
\]
Note that the following rows have a single 1 and all other entries 0: $x_1', x_2', x_3', r_1', r_4'$.
Therefore, by cofactor expansion, the absolute value of the determinant of this matrix is equivalent to the absolute value of the determinant of the following submatrix of the Jacobian:
\[
\begin{array}{c|ccccccc}
&       r_2 &   u_2 \\
\hline
r_2'  & {\scriptstyle \partial f_1 / \partial r_2}   &   {\scriptstyle \partial f_1 / \partial u_2}   \\
r_3'  & {\scriptstyle \partial f_2 / \partial r_2}   &   {\scriptstyle \partial f_2 / \partial u_2}   \\
\end{array}
\]
To compute the Jacobian determinant, it turns out that we only need to compute $4$ of the $49$ elements of the Jacobian.
Also, somewhat surprisingly, we do not need to compute all elements that are not $0$ or $1$.

\section{Automatically Exploiting the Sparse Jacobian Structure}

This section describes how to automatically perform the optimization recognized above in a general setting for arbitrary generative models and RJMCMC kernels, while using automatic differentiation.
Let $t \in \mathbb{R}^K$ denote the vector of values of continuous random choices in the previous model trace.
Let $u \in \mathbb{R}^L$ denote the vector of values of continuous random choices in the proposal trace.
Similarly, let $t' \in \mathbb{R}^{K'}$ and $u' \in \mathbb{R}^{L'}$ denote the vector of values of continuous random choices in the new model trace, and new proposal trace.
We assume that $K + L = K' + L'$ (this is the \emph{dimension matching} requirement), and that we are given a bijection $h : \mathbb{R}^{K + L} \to \mathbb{R}^{K + L}$ that is differentiable and whose Jacobian has nonzero determinant everwhere.\footnote{It may be possible to generalize and allow functions that are not differentiable at a measure zero set of points.}
This section discusses how to efficiently compute the absolute value of the determinant of the Jacobian of $h$, which we denote by $J \in \mathbb{R}^{(K + L) \times (K + L)}$.

$J$ is often sparse due to the fact that often a large fraction of the entries in $(t', u')$ are obtained by \emph{copying} the value of some entry in $(t, u)$.
We denote the subvector of entries of $t'$ that are produced by copying as $t'_c$ and similarly for $u'$ (subvector of copies is denoted $u'_c$).
We denote the subvectors containing other entries of $t'$ and $u'$ by $t'_w$ and $u'_w$ respectively (where $w$ stands for `write').
Let $t_c$ and $u_c$ denote the subvectors of $t$ and $u$ respectively such that each entry is copied to at least one entry in $(t', u')$.
We denote the subvectors containing the remaining entries of $t$ and $u$ as $t_r$ and $u_r$ (where $r$ stands for `read').
The Jacobian of $h$ (up to reordering of rows and columns, which does not change the absolute value of the determinant) then has the following block structure, where white space indicates a zero submatrix (and where sets of rows and columns are labeled on right):
\[
J = \left[
\begin{array}{cccc}
J_{11}  &   J_{12}  &     &   \\
J_{21}  &   J_{22}  &     &   \\
J_{31}  &   J_{32}  &   J_{33}  &   J_{34}\\
J_{41}  &   J_{42}  &   J_{43} &   J_{44}
\end{array}
\right]
\;\;\;\;\;
\begin{array}{c|cccc}
&       t_c    &   u_c &   t_r &   u_r\\
\hline
t'_c  &  J_{11}  &   J_{12}  &     &   \\
u'_c  &  J_{21}  &   J_{22}  &     &   \\
t'_w  &  J_{31}  &   J_{32}  &   J_{33}  &   J_{34}\\
u'_w  &  J_{41}  &   J_{42}  &   J_{43} &   J_{44}
\end{array}
\]

We will show that the absolute value of the determinant of this matrix is the same as the absolute value of the determinant of the following submatrix (which we will show is square), which much smaller than $J$ in the common case when a reversible jump move only modifies a subset of the random choices in a model:
\[
\left[
\begin{array}{cc}
J_{33}  &   J_{34}\\
J_{43} &   J_{44}
\end{array}
\right]
\]

\begin{prop}
The submatrix $[J_{11} \; J_{12}; J_{21} \; J_{22}]$ is square.
\end{prop}
\begin{proof}
Each row of $[J_{11} \; J_{12}; J_{21} \; J_{22}]$ contains exactly one $1$ and all other entries are $0$, by the definition of $t_c'$ and $u_c'$.
Each column of $[J_{11} \; J_{12}; J_{21} \; J_{22}]$ contains at least one $1$, by the definition of $t_c$ and $u_c$.
Therefore, the number of rows must be greater than or equal to the number of columns.
Since we assume that $J$ has nonzero determinant, all its rows must be linearly independent.
Together with the block structure of $J$ shown above, this implies that rows of $[J_{11} \; J_{12}; J_{21} \; J_{22}]$ are linearly independent, and in particular that no two rows are identical.
Since each row of $[J_{11} \; J_{12}; J_{21} \; J_{22}]$ contains exactly one $1$, the number of columns in $[J_{11} \; J_{12}; J_{21} \; J_{22}]$ must be greater than or equal to the number of rows.
\end{proof}

\begin{prop}
The submatrix $[J_{33} \; J_{34}; J_{43} \; J_{44}]$ is square.
\end{prop}
\begin{proof}
$J$ is square and $[J_{11} \; J_{12}; J_{21} \; J_{22}]$ is square.
\end{proof}

\begin{prop}
$|\det(J)| = |\det([J_{33} \; J_{34}; J_{43} \; J_{44}])|$.
\end{prop}
\begin{proof}
Each row of $J$ corresponding to an entry of $t_c'$ or an entry of $u_c'$ has a single $1$ entry and other entries $0$ by definition of $t_c'$ and $u_c'$.
Then, the result follows by consecutive co-factor expansion of the determinant of $J$, deleting all rows in $t_c'$ and all columns in $u_c'$.
\end{proof}

Automating the computation of $|\det(J)|$ by computing instead the determinant of the submatrix allows us to avoid evaluating the entire $J$, which scales quadratically in the maximum number of random choices in the model, and allows us to avoid computing the full determinant, which scales cubically (albeit likely smaller smaller lower-order factors than computing the Jacobian itself).
We only need to evaluate the submatrix $[J_{33} \; J_{34}; J_{43} \; J_{44}]$.
However, this requires knowing which entries in $t$ and $u$ correspond to $(t_r, u_r)$, and the which entries in $t'$ and $u'$ correspond to $(t'_w, u'_w)$.
This is made possible using a domain-specific language (DSL) for defining the bijection $h$ with separate language constructs for:
\begin{enumerate}
\item Reading the values of continuous random choices from $t$ and $u$.
\item Writing the values of continuous random choices to $t'$ and $u'$.
\item Copying the values of continuous random choices from $t$ and $u$ to $t'$ and $u'$.
If the user instead express a copy using a read operation followed by a write operation, but this will result in unecessary extra computation.
\item Tagging read operations for of continuous random choices read from $t$ as \emph{retained}, which asserts that the random choice will not be deleted by the reversible jump move.
If the same address is also not written to in $t'$, and the user assertion is true, then the random choice will be copied implicitly from $t$, which means it can be treated as part of $t_c$, reducing the size of the Jacobian submatrix that needs to be computed.
This is optional -- not tagging a read operation as retained when it is results in unecessary computation.
\end{enumerate}
Automatic differentiation is used to compute the entries of $[J_{33} \; J_{34}; J_{43} \; J_{44}]$.
Finally, note that specification of the involution on discrete random choices may or may not be separated from the specification of the family of bijections $h$ on the continuous random choices.
In the current Gen implementation, a single unified DSL is used to specify the entire involution and the approach described in this document is used for Jacobian determinant correction computation.

\bibliographystyle{unsrt}
\bibliography{references} 


\end{document}
